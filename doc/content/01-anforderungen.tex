\chapter{Anforderungen an das Programm}
\label{ch:anforderungen}

Es soll ein Programm entwickelt werden, das alle Anforderungen abdeckt. Es soll die Möglichkeit geboten werden, einen Startpunkt und bis zu 10 weitere Zieladressen anzugeben, wobei eine \acs{GUI} nicht unbedingt notwendig ist. Bei der Routenplanung geht es dann in erster Linie um die Kosten der Fahrt für einen Elektro-LKW, weniger um die gefahrene Strecke, wobei diese natürlich zuvor in die Berechnung mit einfließt. Dabei gilt der in \cref{tab:energieverbrauch} angegebenen Energieverbrauch.

\begin{table}[H]
	\caption{Der Energieverbrauch des Elektro-LKWs
		\label{tab:energieverbrauch}}
	\begin{center}
		\begin{tabular}{c|c}
			Steigung & Energieverbrauch \\ \hline
			Bergab (Steigung < 3 \%) & 90kw/h pro 100km \\ \hline
			Normal & 100kw/h pro 100km \\ \hline
			Bergauf (Steigung > 3 \%) & 120kw/h pro 100km \\
		\end{tabular}
	\end{center}
\end{table}
